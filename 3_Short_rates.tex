\section{One Factor Short Rate Models}
\subsection{Vasicek Model}
\begin{align*}
dr_t &= k(\theta-r_t)dt + \sigma dW_t\\
r_t &= r_0 e^{-kt} + \theta(1-e^{-kt}) + \sigma e^{-kt} \int_0^t e^{ks} dW_s
\end{align*}
We know that $k$ is speed of mean reversion, 
$\theta$ is the long term mean, and $\sigma$ is the volatility of the short rate.

The drift term $k(\theta-r_t)dt$ is linear in $r_t$, 
and the diffusion term $\sigma dW_t$ is linear in $dW_t$, suggesting 
the rates is normally distributed, meaning it can go negative.\\
To compute the variance:
\begin{align*}
Var(r_t) &= Var\left(\sigma e^{-kt} \int_0^t e^{ks} dW_s\right)\\
&= E\left[\left(\sigma \int_0^t e^{k(t-s)} dW_s\right)^2\right] 
- E\left[\sigma \int_0^t e^{k(t-s)} dW_s\right]^2\\
&= \sigma^2 \int_0^t e^{2k(t-s)} ds - 0\\
&= \frac{\sigma^2}{2k}(1-e^{-2kt})
\end{align*}

So we have $ r_t \sim N\left(A, B\right)$, where
\begin{align*}
    A =& r_0 e^{-kt} + \theta(1-e^{-kt}), \\
    B =& \frac{\sigma^2}{2k}(1-e^{-2kt})
\end{align*}

We can see that as $ t \to \infty$, $ A \to \theta$, 
and $ B \to \frac{\sigma^2}{2k}$,\\
i.e. for larger $k$, the convergence is faster,
and smaller volatiltiy in the long run.\\

More generally, this is a special case of the Ornstein-Uhlenbeck process,
\begin{align*}
dx_t &= \alpha(\mu-x_t)dt + \sigma_t dW_t
\end{align*}
where $x_t$ is the state variable, $\alpha$ is the speed of mean reversion.\\

Some of the problems with this model are:
\begin{itemize}
    \item The short rate can go negative,
    \item Gaussian has thinner tails than market implied distributions, 
    \item The model is endogenous, meaning it cannot reproduce the yield curve.
\end{itemize}
The CIR model is an improvement on the first two points.\\
To solve the third point, we can develop an exogenous model by making
$\theta$ a function of time, $\theta(t)$, i.e.
\begin{align*}
    dr_t &= k(\theta_t-r_t)dt + \sigma dW_t
\end{align*}
So we arrived with the one-factor \textbf{Hull-White} model.\\

\subsection{CIR Model}
For $r_0 > 0$, 
\begin{align*}
dr_t &= k(\theta-r_t)dt + \sigma \sqrt{r_t} dW_t
\end{align*}
ensures positive interest rates with the Feller condition $2k\theta > \sigma^2$, 
(otherwise hits 0).\\
Again we compute the mean and variance:
\begin{align*}
    E[r_t] = r_0 e^{-kt} + \theta(1-e^{-kt})\\
    Var(r_t) = \frac{\sigma^2}{2k}(1-e^{-2kt})\theta + 
    r_0 e^{-kt}\frac{\sigma^2}{k}(1-e^{-kt})
\end{align*}

In this case $r_t$ has a non-central chi-squared distribution, 
has fatter tails than Gaussian, hence closer to market implied distributions.\\
It is less tractable, especially in in multi-factor extensions,
where we need to add correlation between the factors.\\

A problem still, is that Gaussian distributions for the rates are not compatible with the market
implied distributions as they have tails that are too thin.\\

In particular, both models are affine models.
\begin{align*}
    P(t, T) = A(t, T) e^{-B(t, T)r_t}
\end{align*}

Additionally, we work out the continuously compounded spot rate $R(t, T)$,\\
So
\begin{align*}
    R(t, T) = -\frac{\ln P(t, T)}{T-t} \\
    = - \frac{\ln A(t, T)}{T-t} - \frac{\ln e^{-B(t, T)r_t}}{T-t}\\
\end{align*}
So $R$ is an affine transformation of $r_t$.\\
This suggests perfect terminal correlation:
\begin{align*}
    \text{Corr}(R(t, T_1), R(t, T_2)) = 1
\end{align*}

\subsection{Heston Model}
Consider stock price $S_t$ and volatility $v_t$:
\begin{align*}
    dS_t &= r S_t dt + \sqrt{v_t} S_t dW_t^S\\
    dv_t &= \kappa(\theta - v_t) dt + \sigma \sqrt{v_t} dW_t^V
\end{align*}
Here the stock price follows classical GBM, and 
the volatility follows a CIR process. 
This is a stochastic volatility model, where volatility has same assumed
properties as $r_t$ in the CIR model: \\
\begin{itemize}
    \item mean reversion to $\theta$
    \item cannot go negative
    \item has a non-central chi-squared distribution (fatter tails)
\end{itemize}


