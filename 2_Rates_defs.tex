\section{Interest Rate Products}
\subsection{Zero Coupon Bonds}
\begin{definition}
    A \textbf{stochastic discount factor} $D(t, T)$ is the price at time t of receiving 1 at time T.

    Consider a deterministic bond $B_t$ with 
    \begin{equation}
        dB_t = r_t B_t dt
    \end{equation}
    then clearly
    \begin{equation}
        B_t = B_0 \cdot e^{\int_{0}^{t} r_s ds} 
    \end{equation}
    letting $B_T = 1$, we have $ B_0 = e^{-\int_{0}^{T} r_s ds} = D(0, T)$, so the stochastic discount factor is the price of a riskless bond that pays 1 at T.
\end{definition}
\begin{definition}
    A \textbf{Zero Coupon Bond} is a bond that pays its Face Value at T. 
    WLOG assume the face value is 1, then its price at t is given by
    \begin{equation}
        P(t, T) = E_Q\left[ e^{-\int_{t}^{T} r_t dt }\right] = E_Q\left[ D(t, T) \right]
    \end{equation}
    where $r_t$ is the short rate at time t, and the expectation is taken with respect to the risk free measure Q.

    Note in the case when $r_t$ is constant, we have
    \begin{equation}
        P(t, T) = E_Q\left[ e^{-r(T - t)} \right] \\
        = e^{-r(T - t)}
    \end{equation}
\end{definition}

\begin{definition}
    The \textbf{Spot Linear / Simply Compounded / LIBOR Rate} $L$ is the rate that pays linear interest. \\
    Basically when you pay X, you are supposed to receive $X(1 + L \cdot n)$ after n years \\
    To receive 1 at T, while paying $P(t, T)$ at time $t$, so the spot rate $L(t, T)$ is given by
    \begin{align*}
        p(t, T) (1 + L \cdot (T - t)) = 1 \\
        L(t, T) = \frac{\frac{1}{P(t, T)} - 1}{T - t} \\
        = \frac{1 - P(t, T)}{P(t, T) \cdot (T - t)}
    \end{align*}

\end{definition}

\begin{definition}
    The \textbf{Spot Compound / Continuous Rate} $R(t, T)$ is the rate that pays continuous interest\\
    Pay X at t and receive $X \cdot e^{R \cdot (T - t)}$ at T \\
    To receive 1 at T, while paying $P(t, T)$ at time $t$: 
    \begin{align*}
        P(t, T) \cdot e^{R \cdot (T - t)} = 1 
    \end{align*}
    so the spot rate $R(t, T)$ is given by
    $$
        R(t, T) = \frac{\log(1/P)}{T - t} = 
        \frac{-\log P(t, T)}{T - t}
    $$
    Note here we have the short rate $r_t = \lim_{t \to T^-}R(t, T) \approx R(t, t)$
\end{definition}

\begin{definition}
    The \textbf{Annual Spot Rate} $Y$ is the rate that pays interest annually. \\
    Pay X at t and receive $X \cdot (1 + Y)^{T - t}$ at T \\
    We again solve
    \begin{align*}
        P \cdot (1 + Y)^{T - t} = 1
    \end{align*}
    So the spot rate $Y(t, T)$ is given by
    $$
        Y(t, T) = \frac{1}{P^\frac{1}{T - t}} - 1
    $$
    In this case the zero coupon bond price is given by
    $$
        P(t, T) = \frac{1}{(1 + Y)^{T - t}}
    $$
\end{definition}

\begin{definition}
    The \textbf{Zero-Coupon Curve} is the curve given by
$$
    T \mapsto 
\begin{cases} 
    L(t, T), &  T \leq t+1, \\
    Y(t, T), & T > t + 1
\end{cases}
$$
Basically, we use the LIBOR rate for expiry within one year, and the annual spot rate for long term rates.
\end{definition}

\subsection{Forward Rate Agreement}

\begin{definition}
    A \textbf{Forward Rate Agreement} (FRA) is a contract between two parties, where one party pays a fixed interest rate and the other party pays a floating interest rate. The contract is settled at the end of the contract period.
    As we consider at t, the FRA(T, S) is a contract that starts at T and ends at S. The fixed rate is agreed at T, and settled at S. \\
    For the period (T, S), the holder of the \textbf{Payer FRA} pays fixed $K \cdot (S - T)$ and receives floating $L(T, S) \cdot (S - T)$. \\
    The holder of the \textbf{Receiver FRA} has the opposite payoff, with payoff at S given by
    \begin{align*}
        (K - L(T, S)) \cdot (S - T) \cdot N
    \end{align*}
    where N is the notional amount.\\
    We denote $\tau = S - T$.\\
    We discount this back to t with the factor D(t, S), and take the expectation under Q to get the price it at t:
    \begin{align}
        \text{Receiver FRA(t)} =& N \cdot E_Q\left[ D(t, S) \cdot (K - L(T, S)) \cdot (S - T) \right] \notag \\
        =& N \tau \cdot E_Q\left[ D(t, S) \cdot (K - L(T, S)) \right]\\
        =& N (\tau K \cdot P(t, S) - P(t, T) + P(t, S)) 
    \end{align}
\end{definition}
    \begin{proof}
    Note we showed the spot rate $L(T, S)$ is given by
    $$
        L = \frac{1 - P}{P \cdot (T - t)}
    $$
    In this case
    $$
        L(T, S) = \frac{1 - P(T, S)}{P(T, S) \cdot \tau}
    $$
    At S, the Receiver FRA is worth 
    \begin{align*}
        N \cdot \tau \cdot (K - L(T, S))  
        =& N \cdot \tau \cdot (K - \frac{1 - P(T, S)}{P(T, S) \cdot \tau}) \\
        =& N \cdot \tau \cdot (K - \frac{1}{P(T, S) \cdot \tau} + \frac{1}{\tau}) \\
        =& N (\tau K - \frac{1}{P(T, S)} + 1)
    \end{align*}
    Note holding $P(T, S)$ at T is equivalent to holding 1 at S, so we can use $P(T, S)$ as the discount factor from S to T.\\
    Similarly, holding $P(t, T)$ at T is equivalent to holding 1 at T, so we can use $P(t, T) P(T, S)$ as the discount factor from S to t.\\
    At t, the Receiver FRA is multiplied by the discount factor $P(t, S) = P(t, T) P(T, S)$:
    \begin{align*}
        P(t, S) N (\tau K - \frac{1}{P(T, S)} + 1) \\
        = N (\tau K \cdot P(t, S) - P(t, T) + P(t, S)) 
    \end{align*}
\end{proof}

To make the FRA a fair contract, we should make the price of the Receiver FRA equal to 0 at t.\\
We solve FRA(K) = 0 to get the fair rate K:
\begin{align*}
    N (\tau K \cdot P(t, S) - P(t, T) + P(t, S)) = 0 \\
    \Rightarrow K = \frac{P(t, T) - P(t, S)}{\tau \cdot P(t, S)}
\end{align*}

\begin{definition}
    The \textbf{Forward Linear Rate} $F(t, T, S)$ is the rate that pays interest from T to S, agreed at time t. \\
    The forward rate is given by
    \begin{align*}
        F(t, T, S) = \frac{P(t, T) - P(t, S)}{P(t, S) \cdot \tau} = K
    \end{align*}
\end{definition}

Note in (2.1) and (2.2), we showed
\begin{equation}
    \tau \cdot E_Q\left[ D(t, S) \cdot (K - L(T, S)) \right] = \tau K \cdot P(t, S) - P(t, T) + P(t, S) \notag
\end{equation}
Where
\begin{align*}
    E_Q\left[ D(t, S) \cdot (K - L(T, S)) \right] = K \cdot P(t, S) - E_Q\left[ D(t, S) \cdot L(T, S) \right]
\end{align*}
Hence
\begin{align*}
    P(t, T) - P(t, S) =& \tau \big( K \cdot P(t, S) - E_Q\left[ D(t, S) \cdot (K - L) \right] \big)\\
    =& \tau \big( K \cdot P(t, S) - K \cdot P(t, S) + E_Q\left[ D(t, S) \cdot L(T, S) \right] \big)\\
    =& \tau \cdot E_Q\left[ D(t, S) \cdot L(T, S) \right]
\end{align*}
So we have
\begin{align*}
    \frac{P(t, T) - P(t, S)}{\tau} = E_Q\left[ D(t, S) \cdot L(T, S) \right]
\end{align*}
Note the LHS is also equal to $F(t, T, S) \cdot P(t, S)$, so we have
\begin{equation}
    F(t, T, S) \cdot P(t, S) = E_Q\left[ D(t, S) \cdot L(T, S) \right]
\end{equation}
Where we had $P(t, S) = E_Q\left[ D(t, S) \right] $, so we know $ F(t, T, S) \neq E_Q\left[ L(T, S) \right] $.\\
Otherwise we would have $D(t, S)$ and $L(T, S)$ uncorrelated, which is not true.

Note we can rewrite (2.2)
\begin{align}
    \text{Receiver FRA(t)} 
    =& N (\tau K \cdot P(t, S) - P(t, T) + P(t, S)) \notag \\
    =& N \tau \cdot P(t, S) \left( K - \frac{P(t, T)}{P(t, S) \tau} + \frac{1}{\tau}\right) \notag \\  
    =& N \tau \cdot P(t, S) \left( K - \frac{P(t, T) - P(t, S)}{P(t, S) \cdot \tau}\right) \notag \\
    =& N \tau \cdot P(t, S) \left( K - F(t, T, S) \right)
\end{align}
And we obviously have Receiver FRA(K = F(t, T, S)) = 0. \\
Recall that at S, the Receiver FRA is worth 
\begin{align*}
    N \tau (K - L(T, S))  
\end{align*}
So $ F(t, T, S) $ can be viewed as an estimate for the future spot rate $L(T, S)$, which is random at t.

\begin{definition}
    The \textbf{Instantaneous Forward Rate} $f(t, T)$ is the estimated rate to be paid from T to T + dt, 
    computed by taking the following limit \\
    \begin{align*}
        f(t, T) =& \lim_{\tau \to 0} F(t, T, T + \tau) = \lim_{\tau \to 0} \frac{P(t, T) - P(t, T + \tau)}{P(t, T + \tau) \cdot \tau} \\
        =& \lim_{\tau \to 0} -\frac{1}{P(t, T + \tau)} \cdot \frac{P(t, T + \tau) - P(t, T)}{\tau}  \\
        =& -\frac{1}{P(t, T)} \cdot \frac{\partial}{\partial T} P(t, T) \\
        =& -\frac{\partial}{\partial T} \log P(t, T)
    \end{align*}
    This can be viewed as an estimate for the future short rate $r(T)$, which is random at t.
\end{definition}

\subsection{Interest Rate Swaps}
\begin{definition}
    An \textbf{Interest Rate Swap} is a contract between two parties, where FRA are exchanged at future dates $T_{\alpha}, T_{\alpha+1}, \dots, T_{\beta}$.\\
    We define the time intervals $\tau = (\tau_{\alpha+1}, \dots, \tau_{\beta})$, where $\tau_i = T_{i} - T_{i-1}$.\\
    The \textbf{Receiver Swap (RFS)} has payoff at each payment date given by
    \begin{align*}
        N \tau_i (K - L(T_{i-1}, T_i))  
    \end{align*}
    and the  \textbf{Payer Swap (PFS)} has the opposite payoff.\\
    We discount each payment back to t with  $D(t, T_i)$, and take the expectation under Q to get the price it at t:
    \begin{align}
        \text{RFS(t)} =& E_Q\left[ N \sum_{i = \alpha + 1}^{\beta} \tau_i \cdot D(t, T_i) \cdot (K - L(T_{i-1}, T_i)) \right]  
    \end{align}

    We can also view it as a sum of Receiver FRA, so we can plug in (2.4) and have
    \begin{align}
        \text{RFS(t)} =& \sum_{i = \alpha + 1}^{\beta} FRA(t, T_{i-1}, T_i, \tau_i, N) \notag \\
        =& N \sum_{i = \alpha + 1}^{\beta} \tau_i \cdot P(t, T_i) \left( K - F(t, T_{i-1}, T_i) \right) \notag\\
        =& N \sum_{i = \alpha + 1}^{\beta} \tau_i \cdot P(t, T_i)\left( K - \frac{P(t, T_{i-1}) - P(t, T_i)}{P(t, T_i) \cdot \tau_i}\right) \notag\\
        =& N \sum_{i = \alpha + 1}^{\beta} \Big( \tau_i P(t, T_i)  K - P(t, T_{i-1}) + P(t, T_i) \Big) \notag\\
        =& N \Big( \sum_{i = \alpha + 1}^{\beta} \tau_i P(t, T_i)  K - P(t, T_{\alpha}) + P(t, T_{\beta}) \Big) 
    \end{align}
    
    
    
    
    
\end{definition}

Again we plug in RFS(K) = 0 to get the fair rate K, or the \textbf{Forward Swap Rate}:
\begin{align*}
    \sum_{i = \alpha + 1}^{\beta} \tau_i P(t, T_i)  K  = P(t, T_{\alpha}) - P(t, T_{\beta})\\
    \Rightarrow K = \frac{P(t, T_{\alpha}) - P(t, T_{\beta})}{\sum \tau_i \cdot P(t, T_i)}
\end{align*}

\begin{definition}
    The \textbf{Forward Swap Rate} $S_{\alpha, \beta}(t)$ is the rate that makes the interest rate swap a fair contract:
    \begin{align*}
        S(t) = \frac{P(t, T_{\alpha}) - P(t, T_{\beta})}{\sum \tau_i \cdot P(t, T_i)}
    \end{align*}
\end{definition}

Like before, we can rewrite the Receiver Swap (3.2):
\begin{align*}
    \text{RFS(t)} =& N \Big( \sum \tau_i P(t, T_i)  K - \big(P(t, T_{\alpha}) - P(t, T_{\beta})\big) \Big) \notag \\
    =& N  \sum \tau_i P(t, T_i) \Big( K - \frac{P(t, T_{\alpha}) - P(t, T_{\beta})}{\sum \tau_i \cdot P(t, T_i)} \Big) \notag \\
    =& N  \sum \tau_i P(t, T_i) \Big( K - S(t) \Big)
\end{align*}

\subsection{Caps/Floors}
\begin{definition}
    A \textbf{Caplet} is a European call options on interest rate from $T_{i-1}$ to $T_i$, with payoff at $T_i$ given by
    \begin{align*}
        N \cdot \tau_i \left( L(T_{i-1}, T_i) - K \right)^+
    \end{align*}
    And a \textbf{Cap} is simply a sum of Caplets:
    \begin{align*}
        N \sum_{i = \alpha + 1}^{\beta} \tau_i \left( L(T_{i-1}, T_i) - K \right)^+
    \end{align*}
    
    We discount each payment back to t with  $D(t, T_i)$, and take the expectation under Q to price it at t:
    \begin{align}
        \text{Cap(t)} =& E_Q\left[ N \sum_{i = \alpha + 1}^{\beta} \tau_i \cdot D(t, T_i) \cdot \left( L(T_{i-1}, T_i) - K \right)^+ \right]  \\
        =& N \sum_{i = \alpha + 1}^{\beta} \tau_i \cdot P(t, T_i) \cdot Caplet(t, T_{i-1}, T_i, \tau_i, K)
    \end{align}
\end{definition}

To price the Caplet, we need to use Black's formula
\begin{align*}
    Caplet = Black(K, F, \sigma) = F \phi(d_1) - K \phi(d_2)\\
    d_1 = \frac{\text{ln}(F/K) + \frac{\sigma^2 T_{i-1}}{2}}{\sigma \sqrt{T_{i-1}}}\\
    d_2 = \frac{\text{ln}(F/K) - \frac{\sigma^2 T_{i-1}}{2}}{\sigma \sqrt{T_{i-1}}}\\
    \text{Cap(t)} = N \sum_{i = \alpha + 1}^{\beta} \tau_i \cdot P(t, T_i) \cdot Black(K, F(t,T_{i-1}, T_i) , \sigma)
\end{align*}
Here $\sigma$ is the implied volatility retrived from market quotes in [$T_\alpha, T_\beta$], and $F(t,T_{i-1}, T_i)$ is the forward rate.

Similarly for \textbf{Floorlet}, or the put option:
\begin{align*}
    Floorlet = Black(K, F, \sigma) = - F \phi(-d_1) + K \phi(-d_2)
\end{align*}

Recall the standard Black-Scholes:
\begin{align*}
    C =& S_0 N(d_1) - K e^{-rT} N(d_2), \\
    P =& K e^{-rT} N(-d_2) - S_0 N(-d_1),\\
d_1 =& \frac{\ln \left( \frac{S_0}{K} \right) + \left( r + \frac{\sigma^2}{2} \right)T}{\sigma \sqrt{T}},\\
d_2 =& d_1 - \sigma \sqrt{T}.
\end{align*}

\begin{remark}
    A cap is ATM iff its strike is the \textbf{Forward Swap Rate} $S(t)$. \\
    It is ITM if $ K < S(t) $, and OTM if $ K > S(t) $.
\end{remark}
\begin{remark}
    Similar to Call-Put parity, we have
    \begin{align*}
        \text{Caplet}(K) - \text{Floorlet}(K) =& F \phi(d_1) - K \phi(d_2) - ( - F \phi(-d_1) + K \phi(-d_2)) \\
        =& F - K 
    \end{align*}
    Which is the payoff of a single payer FRA\\
    We notice for ATM Caplet and Floorlet with $K = F(t, T_{i-1}, T_i)$, we have $ Caplet= Floorlet $\\
    For ATM Cap and Floor with $ K = S(t) $, we have $ Cap = Floor $ \\
    This is similar to the Call-Put parity, where we have
    \begin{align*}
        C - P = S_0 - K e^{-rT}
    \end{align*}
    And when $K = S_0 e^{rT}$, we have $C = P$.
\end{remark}

\subsection{Swaptions}
To price swaptions, note that the market Black's formula for
caplets/floorlets, consistent with the LMM, is not consistent with the
market Black's formula for swaptions, which is consistent with the
SMM.

To price swaptions under LMM, one has to use Monte Carlo simulations

\subsubsection{Swap Market Model}
SWAPTIONS can be managed well in the LIBOR model only through
approximations like drift freezing. To properly deal with swaptions, one
would have to use the SMM

Recall payoff of swaption at $T_\alpha$ is
\begin{align*}
    (S_{\alpha, \beta} (T_\alpha) - K )^+ \sum_i \tau_i P(T_\alpha, T_i)
\end{align*}

To price it at $ t= 0 $, we add a discount factor $e^{-T_\alpha}$, 
and take risk-neutral expectation:
\begin{align*}
    E_Q \left[ (S_{\alpha, \beta} (T_\alpha) - K )^+ \sum_i \tau_i P(T_\alpha, T_i) 
    \frac{B_0}{B_{T_\alpha}} \right]
\end{align*}

We choose the numeraire $ C_{\alpha, \beta} (t) $ as:
\begin{align*}
    C_{\alpha, \beta} (t) = \sum_i \tau_i P(t, T_i)
\end{align*}

with the related measure be $ Q^{\alpha, \beta}$, under which
any price process divided by $ C_{\alpha, \beta} (t) $ is a martingale

We note that in this case, the forward swap rate
$  S_{\alpha, \beta} (t) $ is a martingale:
\begin{align*}
    S(t) =& \frac{P(t, T_{\alpha}) - P(t, T_{\beta})}{\sum \tau_i \cdot P(t, T_i)}\\
    =& \frac{P(t, T_{\alpha}) - P(t, T_{\beta})}{C_{\alpha, \beta} (t)}
\end{align*}
SO that it has zero drift GBM under $ Q^{\alpha, \beta}$:
\begin{align*}
    dS(t) = \sigma S(t) dW_t^{\alpha, \beta}
\end{align*}
Applying change of numeraire:
\begin{align*}
    E_Q \left[ (S_{\alpha, \beta} (T_\alpha) - K )^+ C_{\alpha, \beta} (T_\alpha) 
    \frac{B_0}{B_{T_\alpha}} \right] 
    =& E_{Q_{\alpha, \beta}} \left[ (S_{\alpha, \beta} (T_\alpha) - K )^+ C_{\alpha, \beta} (T_\alpha) 
    \frac{C_0}{C_{\alpha, \beta} (T_\alpha)} \right]\\
    =& C_{\alpha, \beta} (0) E_{Q_{\alpha, \beta}} \left[ (S_{\alpha, \beta} (T_\alpha) - K )^+ \right]
\end{align*}

Under this setting, we can simply apply Black's formula:
\begin{align*}
    Swaption(0) = Black(S, F, \sigma) = 
    C_{\alpha, \beta}(0) \bigg[S_{\alpha, \beta}(0) \phi(d_1) - K \phi(d_2)\bigg]\\
    = \sum_i \tau_i P(0, T_i) \bigg(S_{\alpha, \beta}(0) \phi(d_1) - K \phi(d_2)\bigg)
\end{align*}

\begin{remark}
    SMM is the only model that is consistent with this market formula.

    LMM is not compatible with the Black formula for Swaptions.
\end{remark}

\subsection{Swaption Under LMM}
